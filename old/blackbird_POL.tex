% Options for packages loaded elsewhere
\PassOptionsToPackage{unicode}{hyperref}
\PassOptionsToPackage{hyphens}{url}
%
\documentclass[
]{article}
\usepackage{amsmath,amssymb}
\usepackage{lmodern}
\usepackage{iftex}
\ifPDFTeX
  \usepackage[T1]{fontenc}
  \usepackage[utf8]{inputenc}
  \usepackage{textcomp} % provide euro and other symbols
\else % if luatex or xetex
  \usepackage{unicode-math}
  \defaultfontfeatures{Scale=MatchLowercase}
  \defaultfontfeatures[\rmfamily]{Ligatures=TeX,Scale=1}
\fi
% Use upquote if available, for straight quotes in verbatim environments
\IfFileExists{upquote.sty}{\usepackage{upquote}}{}
\IfFileExists{microtype.sty}{% use microtype if available
  \usepackage[]{microtype}
  \UseMicrotypeSet[protrusion]{basicmath} % disable protrusion for tt fonts
}{}
\makeatletter
\@ifundefined{KOMAClassName}{% if non-KOMA class
  \IfFileExists{parskip.sty}{%
    \usepackage{parskip}
  }{% else
    \setlength{\parindent}{0pt}
    \setlength{\parskip}{6pt plus 2pt minus 1pt}}
}{% if KOMA class
  \KOMAoptions{parskip=half}}
\makeatother
\usepackage{xcolor}
\usepackage[margin=1in]{geometry}
\usepackage{color}
\usepackage{fancyvrb}
\newcommand{\VerbBar}{|}
\newcommand{\VERB}{\Verb[commandchars=\\\{\}]}
\DefineVerbatimEnvironment{Highlighting}{Verbatim}{commandchars=\\\{\}}
% Add ',fontsize=\small' for more characters per line
\usepackage{framed}
\definecolor{shadecolor}{RGB}{248,248,248}
\newenvironment{Shaded}{\begin{snugshade}}{\end{snugshade}}
\newcommand{\AlertTok}[1]{\textcolor[rgb]{0.94,0.16,0.16}{#1}}
\newcommand{\AnnotationTok}[1]{\textcolor[rgb]{0.56,0.35,0.01}{\textbf{\textit{#1}}}}
\newcommand{\AttributeTok}[1]{\textcolor[rgb]{0.77,0.63,0.00}{#1}}
\newcommand{\BaseNTok}[1]{\textcolor[rgb]{0.00,0.00,0.81}{#1}}
\newcommand{\BuiltInTok}[1]{#1}
\newcommand{\CharTok}[1]{\textcolor[rgb]{0.31,0.60,0.02}{#1}}
\newcommand{\CommentTok}[1]{\textcolor[rgb]{0.56,0.35,0.01}{\textit{#1}}}
\newcommand{\CommentVarTok}[1]{\textcolor[rgb]{0.56,0.35,0.01}{\textbf{\textit{#1}}}}
\newcommand{\ConstantTok}[1]{\textcolor[rgb]{0.00,0.00,0.00}{#1}}
\newcommand{\ControlFlowTok}[1]{\textcolor[rgb]{0.13,0.29,0.53}{\textbf{#1}}}
\newcommand{\DataTypeTok}[1]{\textcolor[rgb]{0.13,0.29,0.53}{#1}}
\newcommand{\DecValTok}[1]{\textcolor[rgb]{0.00,0.00,0.81}{#1}}
\newcommand{\DocumentationTok}[1]{\textcolor[rgb]{0.56,0.35,0.01}{\textbf{\textit{#1}}}}
\newcommand{\ErrorTok}[1]{\textcolor[rgb]{0.64,0.00,0.00}{\textbf{#1}}}
\newcommand{\ExtensionTok}[1]{#1}
\newcommand{\FloatTok}[1]{\textcolor[rgb]{0.00,0.00,0.81}{#1}}
\newcommand{\FunctionTok}[1]{\textcolor[rgb]{0.00,0.00,0.00}{#1}}
\newcommand{\ImportTok}[1]{#1}
\newcommand{\InformationTok}[1]{\textcolor[rgb]{0.56,0.35,0.01}{\textbf{\textit{#1}}}}
\newcommand{\KeywordTok}[1]{\textcolor[rgb]{0.13,0.29,0.53}{\textbf{#1}}}
\newcommand{\NormalTok}[1]{#1}
\newcommand{\OperatorTok}[1]{\textcolor[rgb]{0.81,0.36,0.00}{\textbf{#1}}}
\newcommand{\OtherTok}[1]{\textcolor[rgb]{0.56,0.35,0.01}{#1}}
\newcommand{\PreprocessorTok}[1]{\textcolor[rgb]{0.56,0.35,0.01}{\textit{#1}}}
\newcommand{\RegionMarkerTok}[1]{#1}
\newcommand{\SpecialCharTok}[1]{\textcolor[rgb]{0.00,0.00,0.00}{#1}}
\newcommand{\SpecialStringTok}[1]{\textcolor[rgb]{0.31,0.60,0.02}{#1}}
\newcommand{\StringTok}[1]{\textcolor[rgb]{0.31,0.60,0.02}{#1}}
\newcommand{\VariableTok}[1]{\textcolor[rgb]{0.00,0.00,0.00}{#1}}
\newcommand{\VerbatimStringTok}[1]{\textcolor[rgb]{0.31,0.60,0.02}{#1}}
\newcommand{\WarningTok}[1]{\textcolor[rgb]{0.56,0.35,0.01}{\textbf{\textit{#1}}}}
\usepackage{graphicx}
\makeatletter
\def\maxwidth{\ifdim\Gin@nat@width>\linewidth\linewidth\else\Gin@nat@width\fi}
\def\maxheight{\ifdim\Gin@nat@height>\textheight\textheight\else\Gin@nat@height\fi}
\makeatother
% Scale images if necessary, so that they will not overflow the page
% margins by default, and it is still possible to overwrite the defaults
% using explicit options in \includegraphics[width, height, ...]{}
\setkeys{Gin}{width=\maxwidth,height=\maxheight,keepaspectratio}
% Set default figure placement to htbp
\makeatletter
\def\fps@figure{htbp}
\makeatother
\setlength{\emergencystretch}{3em} % prevent overfull lines
\providecommand{\tightlist}{%
  \setlength{\itemsep}{0pt}\setlength{\parskip}{0pt}}
\setcounter{secnumdepth}{-\maxdimen} % remove section numbering
\ifLuaTeX
  \usepackage{selnolig}  % disable illegal ligatures
\fi
\IfFileExists{bookmark.sty}{\usepackage{bookmark}}{\usepackage{hyperref}}
\IfFileExists{xurl.sty}{\usepackage{xurl}}{} % add URL line breaks if available
\urlstyle{same} % disable monospaced font for URLs
\hypersetup{
  pdftitle={The Pace of Life of Partial Migration},
  pdfauthor={Scott Yanco, Nils Linek, Micheal Kearney, \ldots, Martin Wikelski, Jesko Partecke},
  hidelinks,
  pdfcreator={LaTeX via pandoc}}

\title{The Pace of Life of Partial Migration}
\author{Scott Yanco, Nils Linek, Micheal Kearney, \ldots, Martin
Wikelski, Jesko Partecke}
\date{2022-07-20}

\begin{document}
\maketitle

\hypertarget{todo}{%
\subsection{TODO:}\label{todo}}

\begin{itemize}
\tightlist
\item
  @Nils: fill in field methods, study area, data collection, data
  processing(?), etc\ldots{}
\end{itemize}

\hypertarget{introduction}{%
\subsection{Introduction}\label{introduction}}

\begin{itemize}
\tightlist
\item
  Life history is fundamental to ecology and evolution, blah blah.

  \begin{itemize}
  \tightlist
  \item
    Because life history, in its essence, is about the allocation of
    finite energy over finite time horizons, the POL theory posits that
    life history trade offs are physiologically/metabolically mediated
  \item
    Examining the metabolic underpinning of POL has the added benefit of
    allowing us to measure the \emph{strategy} rather than the outcome
    (e.g., a life table).
  \end{itemize}
\item
  Understanding if and how life history relates to behavioral phenotypes
  remains a key challenge in evolutionary ecology.

  \begin{itemize}
  \tightlist
  \item
    Such an understanding would help us make better predictions about
    the persistence or vulnerability of particular phenotypes under
    environmental change.
  \item
    Furthermore, understanding how certain behavior types generate
    fitness promises to clarify how and why such behaviors emerge in the
    first place by describing both the ecological context and the
    evolutionary reward associated with such a behavior. -Some good
    examples with movement:

    \begin{itemize}
    \tightlist
    \item
      Campos Candela Ecol letters
    \item
      Orr Spiegel sleepy lizards
    \item
      The Corsican Blue Tit stuff
    \end{itemize}
  \end{itemize}
\item
  Seasonal migration is one such behavior whose eco-evolutionary cause
  remains only poorly understood.

  \begin{itemize}
  \tightlist
  \item
    Review relevant lit here: Winger's ``Red Queen'', Zink papers, Yanco
    et al 2021, etc.
  \item
    Futhermore, whether migratory behavior is consistently associated
    with a fast or slow pace of life is also debated:

    \begin{itemize}
    \tightlist
    \item
      Soriano-Redondo
    \item
      Winger and Pegan
    \item
      Wikelski stonechats
    \item
      Yanco and Pierce, Anderson and Jetz (not mig specific)
    \end{itemize}
  \end{itemize}
\item
  Therefore here we examine the metabolic dynamics across a full annual
  cycle for a partially migratory population of Blackbirds (\emph{Turdus
  merla}). After controlling for differences in the metabolic cost of
  maintaining thermal homeostasis we examine individual metabolic at
  daily resolution for individuals that reside year-round in southern
  Germany and compare theese to birds that breed sympatrically with the
  resident population but overwinter in southern France.
\end{itemize}

\hypertarget{methodsresults}{%
\subsection{Methods/Results}\label{methodsresults}}

\begin{Shaded}
\begin{Highlighting}[]
\CommentTok{\# Libraries}
\FunctionTok{library}\NormalTok{(tidyverse)}
\end{Highlighting}
\end{Shaded}

\begin{verbatim}
## -- Attaching packages --------------------------------------- tidyverse 1.3.2 --
## v ggplot2 3.3.6     v purrr   0.3.4
## v tibble  3.1.7     v dplyr   1.0.9
## v tidyr   1.2.0     v stringr 1.4.0
## v readr   2.1.2     v forcats 0.5.1
## -- Conflicts ------------------------------------------ tidyverse_conflicts() --
## x dplyr::filter() masks stats::filter()
## x dplyr::lag()    masks stats::lag()
\end{verbatim}

\begin{Shaded}
\begin{Highlighting}[]
\FunctionTok{library}\NormalTok{(ggplot2)}
\FunctionTok{library}\NormalTok{(mgcv)}
\end{Highlighting}
\end{Shaded}

\begin{verbatim}
## Loading required package: nlme
## 
## Attaching package: 'nlme'
## 
## The following object is masked from 'package:dplyr':
## 
##     collapse
## 
## This is mgcv 1.8-40. For overview type 'help("mgcv-package")'.
\end{verbatim}

\begin{Shaded}
\begin{Highlighting}[]
\FunctionTok{library}\NormalTok{(patchwork)}

\CommentTok{\# Color Pallette}
\NormalTok{pal }\OtherTok{\textless{}{-}} \FunctionTok{c}\NormalTok{(}\StringTok{"ws"} \OtherTok{=} \StringTok{"\#B1624EFF"}\NormalTok{, }\StringTok{"fm"} \OtherTok{=} \StringTok{"\#5CC8D7FF"}\NormalTok{) }
\end{Highlighting}
\end{Shaded}

\hypertarget{study-area}{%
\subsubsection{Study area}\label{study-area}}

Germany\ldots{}

\hypertarget{capture}{%
\subsubsection{Capture}\label{capture}}

Nils lived in a van and captured many birds\ldots{}

\hypertarget{biologgers}{%
\subsubsection{Biologgers}\label{biologgers}}

Nils did many bird surgeries\ldots{}

\hypertarget{full-annual-cycle-metabolism}{%
\subsubsection{Full annual cycle
metabolism}\label{full-annual-cycle-metabolism}}

In order to compare metabolism between migrant and non-migrant
blackbirds, we estimated daily metabolic rate for 61 individuals (17
migrants and 45 non-migrants) \textbf{why don't these numbers sum
correctly\ldots?} for 2-210 days spanning fall capture dates to spring
recaptures (mean 178.57 days). We estimated daily metabolic rate as a
function of the observed average daily heart rate. To convert heart rate
(in units of beats-per-minute) to Watts we first needed to estimate a
function relating the two quantities. To do this, we subset the data to
only non-migrant individuals and only included nocturnal periods of
observation. Thus, this reduced data set only includes observations
wherein metabolic activity (and heart rate) are expected to be related
to basal metabolic rate plus thermoregulation. We then estimated the
corresponding metabolic rate for these observations using the endotherm
model (\texttt{endoR} function) from the \texttt{nichemapr} R package
(Kearney XXXX). We supplied the model with observed ambient temperatures
derived from {[}Nils insert here{]} as well as the observed body
temperature recorded by the biologgers. The full set of input parameters
for this model is included as Appendix A.

\begin{Shaded}
\begin{Highlighting}[]
\CommentTok{\# NOT RUN}
\ExtensionTok{conda}\NormalTok{ activate nichemapr}
\ExtensionTok{RScript}\NormalTok{ \textasciitilde{}/projects/blackbird\_POL/src/workflow/night\_ind.r}
\end{Highlighting}
\end{Shaded}

To estimate the functional relationship between the predicted metabolic
output and observed heart rate we fit a generalized additive model (GAM)
using the \texttt{gam} function from the \texttt{mgcv} R package (REF).
The model included a smooth effect of Julian day to account for seasonal
differences in e.g., body size, tissue anabolism/catabolism, etc. and a
fixed effect of heart rate. To account for individual heterogeneity we
included a random slope for heart rate nested within individual. We then
used the \texttt{predict} function (excluding random effect variance) to
estimate predicted metabolic rate from the full heart rate dataset,
resulting in full annual metabolic curves for all individuals (Fig 2).

\begin{Shaded}
\begin{Highlighting}[]
\CommentTok{\#observed met from hr}
\NormalTok{nd }\OtherTok{\textless{}{-}}\NormalTok{ tot\_out\_df }\SpecialCharTok{\%\textgreater{}\%} 
  \CommentTok{\# filter(strat == "ws") \%\textgreater{}\% }
  \CommentTok{\# select(heartrate, juian.bird) \%\textgreater{}\% }
  \FunctionTok{rename}\NormalTok{(}\AttributeTok{hrt =}\NormalTok{ heartrate) }

\NormalTok{p }\OtherTok{\textless{}{-}} \FunctionTok{predict}\NormalTok{(fm.m,}
                \AttributeTok{newdata =}\NormalTok{ nd,}
                \AttributeTok{exclude =} \FunctionTok{c}\NormalTok{(}\StringTok{"s(logger.id,hrt)"}\NormalTok{,}\StringTok{"s(logger.id)"}\NormalTok{))}

\NormalTok{pred }\OtherTok{\textless{}{-}} \FunctionTok{cbind}\NormalTok{(nd, p) }\SpecialCharTok{\%\textgreater{}\%} 
  \FunctionTok{mutate}\NormalTok{(}\AttributeTok{tot\_min\_therm =}\NormalTok{ p}\SpecialCharTok{{-}}\NormalTok{metab,}
         \AttributeTok{jbf =} \FunctionTok{as.factor}\NormalTok{(julian.bird)) }

\CommentTok{\#summarize across individuals (within each strategy{-}day)}
\NormalTok{pred.sum }\OtherTok{\textless{}{-}}\NormalTok{ pred }\SpecialCharTok{\%\textgreater{}\%} 
  \FunctionTok{group\_by}\NormalTok{(strat, julian.bird) }\SpecialCharTok{\%\textgreater{}\%} 
  \FunctionTok{summarize}\NormalTok{(}\AttributeTok{mhrt =} \FunctionTok{mean}\NormalTok{(hrt),}
          \AttributeTok{mp =} \FunctionTok{mean}\NormalTok{(p),}
          \AttributeTok{mmet =} \FunctionTok{mean}\NormalTok{(metab),}
          \AttributeTok{mtemp =} \FunctionTok{mean}\NormalTok{(temp),}
          \AttributeTok{mdiff =} \FunctionTok{mean}\NormalTok{(tot\_min\_therm))}
\end{Highlighting}
\end{Shaded}

\begin{verbatim}
## `summarise()` has grouped output by 'strat'. You can override using the
## `.groups` argument.
\end{verbatim}

\begin{Shaded}
\begin{Highlighting}[]
\FunctionTok{ggplot}\NormalTok{(pred) }\SpecialCharTok{+} 
  \FunctionTok{geom\_line}\NormalTok{(}\FunctionTok{aes}\NormalTok{(}\AttributeTok{x=}\NormalTok{julian.bird, }\AttributeTok{y=}\NormalTok{p, }\AttributeTok{color =}\NormalTok{ strat, }\AttributeTok{group =}\NormalTok{ band), }\AttributeTok{alpha =} \FloatTok{0.5}\NormalTok{) }\SpecialCharTok{+}
  \FunctionTok{scale\_color\_manual}\NormalTok{(}\AttributeTok{values =}\NormalTok{ pal, }\AttributeTok{labels =} \FunctionTok{c}\NormalTok{(}\StringTok{"Residents"}\NormalTok{, }\StringTok{"Migrants"}\NormalTok{),}
                     \AttributeTok{name =} \StringTok{"Migratory Strategy"}\NormalTok{) }\SpecialCharTok{+}
  \FunctionTok{ylab}\NormalTok{(}\StringTok{"Metabolic Rate (W)"}\NormalTok{) }\SpecialCharTok{+}
  \FunctionTok{xlab}\NormalTok{(}\StringTok{"Julian Bird"}\NormalTok{) }\SpecialCharTok{+}
  \FunctionTok{theme\_minimal}\NormalTok{()}
\end{Highlighting}
\end{Shaded}

\begin{figure}
\centering
\includegraphics{blackbird_POL_files/figure-latex/unnamed-chunk-7-1.pdf}
\caption{Figure 2. Estimated metabolic rate for 61 blackbirds}
\end{figure}

\begin{Shaded}
\begin{Highlighting}[]
\FunctionTok{ggplot}\NormalTok{(pred) }\SpecialCharTok{+} 
  \FunctionTok{geom\_boxplot}\NormalTok{(}\FunctionTok{aes}\NormalTok{(}\AttributeTok{x=}\FunctionTok{as.factor}\NormalTok{(julian.bird), }\AttributeTok{y=}\NormalTok{p, }\AttributeTok{color =}\NormalTok{ strat)) }\SpecialCharTok{+}
  \FunctionTok{scale\_color\_manual}\NormalTok{(}\AttributeTok{values =}\NormalTok{ pal, }\AttributeTok{labels =} \FunctionTok{c}\NormalTok{(}\StringTok{"Residents"}\NormalTok{, }\StringTok{"Migrants"}\NormalTok{),}
                     \AttributeTok{name =} \StringTok{"Migratory Strategy"}\NormalTok{) }\SpecialCharTok{+}
  \FunctionTok{ylab}\NormalTok{(}\StringTok{"Metabolic Rate (W)"}\NormalTok{) }\SpecialCharTok{+}
  \FunctionTok{xlab}\NormalTok{(}\StringTok{"Julian Bird"}\NormalTok{) }\SpecialCharTok{+}
  \FunctionTok{theme\_minimal}\NormalTok{()}
\end{Highlighting}
\end{Shaded}

\includegraphics{blackbird_POL_files/figure-latex/unnamed-chunk-8-1.pdf}
This analysis reveals no significant difference in total metabolic
expenditure between the two migratory strategies. \textbf{{[}confirm
with Nils that I'm analyzing this correctly - currently just visually
assessing overlap in box plots. The approach here should match what he's
doing with heart rate in his other paper.{]}} However, because migrants
and residents experience different weather regimes during migration and
over-wintering, metabolic expenditures for thermoregulation may differ
between the strategies. Moreover, differences in thermoregulatory
expenditures between the strategies could result in differences in
``metabolic remainder: (i.e., the metabolic activity after controlling
for basal metabolic rate and thermoregulation) and thus mask possible
differences in POL. Therefore we controlled for potential differences in
thermoregulatory metabolism by again using the \texttt{endoR} function
of the \texttt{nichemapr} package to estimate basal metabolic rate +
thermoregulatory metabolism based on the observed ambient and body
temperatures for the 61 blackbirds. Input parameters matched those
supplied to the nighttime-only analysis (Appendix A) with the exception
of solar irradiation. We estimated solar irradiation using the
\texttt{micro\_global} function from the \texttt{nichemapr} package
which allows us to estimate daily solar irradiation for any location on
earth. We used the approximate location of the shared breeding grounds
(8.96691, 47.745237) for all individuals. The differing weather contexts
during the wintering period resulted in markedly different metabolic
costs of thermoregulation between the two strategies, with migrants
expending substantially less energy (Fig. 3A) to maintain approximately
equivalent body temperatures (Fig. 3B) \textbf{{[}@Nils we could model
this directly but I can't remember if you're already doing this as part
of you current paper - if you are we should just reference that here{]}}
\textbf{{[}NB for the solar irradiation, we are currently using
Radolfzell as the location for all birds - we probably want to match the
same approach we use for temperature here; that said. this approach is
conservative, so once we give the migrants even more solar energy, it
will further reduce the metabolic costs of thermoregulation and thuis
magnify the differences in the ``metabolic remainder''\ldots{]}}

\begin{Shaded}
\begin{Highlighting}[]
\CommentTok{\# NOT RUN}
\ExtensionTok{conda}\NormalTok{ activate nichemapr}
\ExtensionTok{RScript}\NormalTok{ \textasciitilde{}/projects/blackbird\_POL/src/workflow/met\_model\_full.r}
\end{Highlighting}
\end{Shaded}

\begin{Shaded}
\begin{Highlighting}[]
\NormalTok{p1 }\OtherTok{\textless{}{-}} \FunctionTok{ggplot}\NormalTok{(pred) }\SpecialCharTok{+} 
  \FunctionTok{geom\_boxplot}\NormalTok{(}\FunctionTok{aes}\NormalTok{(}\AttributeTok{x=}\FunctionTok{as.factor}\NormalTok{(julian.bird), }\AttributeTok{y=}\NormalTok{metab, }\AttributeTok{color =}\NormalTok{ strat)) }\SpecialCharTok{+}
  \FunctionTok{scale\_color\_manual}\NormalTok{(}\AttributeTok{values =}\NormalTok{ pal, }\AttributeTok{labels =} \FunctionTok{c}\NormalTok{(}\StringTok{"Residents"}\NormalTok{, }\StringTok{"Migrants"}\NormalTok{),}
                     \AttributeTok{name =} \StringTok{"Migratory Strategy"}\NormalTok{) }\SpecialCharTok{+}
  \FunctionTok{ylab}\NormalTok{(}\StringTok{"Thermoregulatory }\SpecialCharTok{\textbackslash{}n}\StringTok{ Metabolic Rate (W)"}\NormalTok{) }\SpecialCharTok{+}
  \FunctionTok{xlab}\NormalTok{(}\StringTok{"Julian Bird"}\NormalTok{) }\SpecialCharTok{+}
  \FunctionTok{theme\_minimal}\NormalTok{()}

\NormalTok{p2 }\OtherTok{\textless{}{-}} \FunctionTok{ggplot}\NormalTok{(pred) }\SpecialCharTok{+} 
  \FunctionTok{geom\_boxplot}\NormalTok{(}\FunctionTok{aes}\NormalTok{(}\AttributeTok{x=}\FunctionTok{as.factor}\NormalTok{(julian.bird), }\AttributeTok{y=}\NormalTok{temp, }\AttributeTok{color =}\NormalTok{ strat)) }\SpecialCharTok{+}
  \FunctionTok{scale\_color\_manual}\NormalTok{(}\AttributeTok{values =}\NormalTok{ pal, }\AttributeTok{labels =} \FunctionTok{c}\NormalTok{(}\StringTok{"Residents"}\NormalTok{, }\StringTok{"Migrants"}\NormalTok{),}
                     \AttributeTok{name =} \StringTok{"Migratory Strategy"}\NormalTok{) }\SpecialCharTok{+}
  \FunctionTok{ylab}\NormalTok{(}\StringTok{"Body Temperature (C)"}\NormalTok{) }\SpecialCharTok{+}
  \FunctionTok{xlab}\NormalTok{(}\StringTok{"Julian Bird"}\NormalTok{) }\SpecialCharTok{+}
  \FunctionTok{theme\_minimal}\NormalTok{() }\SpecialCharTok{+}
  \FunctionTok{theme}\NormalTok{(}\AttributeTok{legend.position =} \StringTok{"none"}\NormalTok{)}

\NormalTok{p1 }\SpecialCharTok{/}\NormalTok{ p2 }\SpecialCharTok{+} \FunctionTok{plot\_annotation}\NormalTok{(}\AttributeTok{tag\_levels =} \StringTok{\textquotesingle{}A\textquotesingle{}}\NormalTok{)}
\end{Highlighting}
\end{Shaded}

\begin{figure}
\centering
\includegraphics{blackbird_POL_files/figure-latex/unnamed-chunk-10-1.pdf}
\caption{Fig. 3. A) Daily metabolic rate for basal metabolism and
thermoregulation only for both migrant and resident blackbirds; B) Daily
observed body temperatures for both migrant and resident blackbirds}
\end{figure}

To calculate the daily individual ``metabolic remainder'', for each
individual we subtracted the daily estimated metabolic expenditure for
basal metabolism + thermoregulation from the estimated total daily
metabolic rate. This metabolic remainder, then, describes the metabolic
activity directed toward all other biological processes after
controlling for basal metabolic rates and differential thermoregulatory
costs between the two strategies.

\begin{Shaded}
\begin{Highlighting}[]
\FunctionTok{ggplot}\NormalTok{(pred)}\SpecialCharTok{+}
  \FunctionTok{geom\_boxplot}\NormalTok{(}\FunctionTok{aes}\NormalTok{(}\AttributeTok{x =}\NormalTok{ jbf, }\AttributeTok{y =}\NormalTok{ tot\_min\_therm, }\AttributeTok{color =}\NormalTok{ strat )) }\SpecialCharTok{+}
  \FunctionTok{scale\_color\_manual}\NormalTok{(}\AttributeTok{values =}\NormalTok{ pal, }\AttributeTok{labels =} \FunctionTok{c}\NormalTok{(}\StringTok{"Residents"}\NormalTok{, }\StringTok{"Migrants"}\NormalTok{),}
                     \AttributeTok{name =} \StringTok{"Migratory Strategy"}\NormalTok{) }\SpecialCharTok{+}
  \FunctionTok{ylab}\NormalTok{(}\StringTok{"Metabolic Remainder (W)"}\NormalTok{) }\SpecialCharTok{+}
  \FunctionTok{xlab}\NormalTok{(}\StringTok{"Julian Bird"}\NormalTok{) }\SpecialCharTok{+}
  \FunctionTok{theme\_minimal}\NormalTok{()}
\end{Highlighting}
\end{Shaded}

\begin{figure}
\centering
\includegraphics{blackbird_POL_files/figure-latex/unnamed-chunk-11-1.pdf}
\caption{Fig. 4. Boxplot of daily metabolic remainder for migrant and
resident blackbirds. metabolic remainder is the estimated daily
metabolic expenditure after removing basal metabolism and
themoregulatory expenses.}
\end{figure}

We find significant differences in the metabolic remainder across the
two migratory strategies (Fig. 4. Specifically, between days XX and XX
\textbf{{[}find a good way to calc this\ldots{]}} migrants display
substantially higher metabolic remainder (2-3X the metabolic remainder
observed in year-round residents) suggesting a faster POL.

\textbf{{[}model this? could fit a gam with smooth by JD interacting tih
strategy, RE of individual{]}}

We also considered whether the observed differences in metabolic
remainder resulted in differences in body condition at the start of the
breeding season. To do this, we subset our data to only include
observations following the onset of spring arrival of migrants. We
calculated body condition as
\(\text{body mass (g)}\text{tarsus length (mm)}\). We modeled body
condition using the \texttt{gam} function in \texttt{mgcv} and
considered it a smooth function of Julian date (to account for
phenological effects on body mass), and a fixed effect of the
interaction between sex and migratory strategy. We accounted for
individual variation by including a random intercept by individual.

\textbf{{[}@Nils, should we right truncate too? i.e., are there some
observations that are too far into the breeding season to be useful?
Also, can we better control for days since arrival than just raw Julian
date?{]}}

\begin{Shaded}
\begin{Highlighting}[]
\NormalTok{bodymass1 }\OtherTok{\textless{}{-}}\NormalTok{ bodymass }\SpecialCharTok{\%\textgreater{}\%} 
  \FunctionTok{left\_join}\NormalTok{(tarsus) }\SpecialCharTok{\%\textgreater{}\%} 
  \FunctionTok{mutate}\NormalTok{(}\AttributeTok{period =} \FunctionTok{case\_when}\NormalTok{(julian.bird }\SpecialCharTok{\textless{}} \DecValTok{150} \SpecialCharTok{\textasciitilde{}} \StringTok{"pre"}\NormalTok{,}
\NormalTok{                   julian.bird }\SpecialCharTok{\textgreater{}=} \DecValTok{150} \SpecialCharTok{\textasciitilde{}} \StringTok{"post"}\NormalTok{)) }\SpecialCharTok{\%\textgreater{}\%} 
  \FunctionTok{mutate}\NormalTok{(}\AttributeTok{birdid.fk =} \FunctionTok{factor}\NormalTok{(birdid.fk),}
         \AttributeTok{bc =}\NormalTok{ bodymass}\SpecialCharTok{/}\NormalTok{tarsus) }\SpecialCharTok{\%\textgreater{}\%} 
  \FunctionTok{filter}\NormalTok{(period }\SpecialCharTok{==} \StringTok{"post"}\NormalTok{)}
\end{Highlighting}
\end{Shaded}

\begin{verbatim}
## Joining, by = c("birdid.fk", "non.breeding.strategy", "sex", "julian.bird")
\end{verbatim}

\begin{Shaded}
\begin{Highlighting}[]
\NormalTok{fm.bm }\OtherTok{\textless{}{-}}\FunctionTok{gam}\NormalTok{(bc }\SpecialCharTok{\textasciitilde{}} \FunctionTok{s}\NormalTok{(julian.bird) }\SpecialCharTok{+}\NormalTok{ non.breeding.strategy}\SpecialCharTok{*}\NormalTok{sex }\SpecialCharTok{+} \FunctionTok{s}\NormalTok{(birdid.fk, }\AttributeTok{bs =} \StringTok{"re"}\NormalTok{), }\AttributeTok{data =}\NormalTok{ bodymass1, }\AttributeTok{method =} \StringTok{"REML"}\NormalTok{)}
\FunctionTok{summary}\NormalTok{(fm.bm)}
\end{Highlighting}
\end{Shaded}

\begin{verbatim}
## 
## Family: gaussian 
## Link function: identity 
## 
## Formula:
## bc ~ s(julian.bird) + non.breeding.strategy * sex + s(birdid.fk, 
##     bs = "re")
## 
## Parametric coefficients:
##                                 Estimate Std. Error t value Pr(>|t|)    
## (Intercept)                      2.66553    0.04500  59.236   <2e-16 ***
## non.breeding.strategyws          0.02509    0.05493   0.457    0.649    
## sexMale                         -0.05342    0.07295  -0.732    0.467    
## non.breeding.strategyws:sexMale -0.11485    0.08531  -1.346    0.183    
## ---
## Signif. codes:  0 '***' 0.001 '**' 0.01 '*' 0.05 '.' 0.1 ' ' 1
## 
## Approximate significance of smooth terms:
##                  edf Ref.df     F p-value   
## s(julian.bird) 6.074  7.177 3.625  0.0025 **
## s(birdid.fk)   2.762 56.000 0.052  0.3830   
## ---
## Signif. codes:  0 '***' 0.001 '**' 0.01 '*' 0.05 '.' 0.1 ' ' 1
## 
## R-sq.(adj) =  0.363   Deviance explained = 46.6%
## -REML = -19.041  Scale est. = 0.02088   n = 74
\end{verbatim}

\begin{Shaded}
\begin{Highlighting}[]
\CommentTok{\# calc cis using t distribution and SEs from the VCOV}
\NormalTok{Vb }\OtherTok{\textless{}{-}} \FunctionTok{vcov}\NormalTok{(fm.bm, }\AttributeTok{unconditional =} \ConstantTok{TRUE}\NormalTok{)}
\NormalTok{se }\OtherTok{\textless{}{-}} \FunctionTok{sqrt}\NormalTok{(}\FunctionTok{diag}\NormalTok{(Vb))}

\CommentTok{\# rdf \textless{}{-} df.residual(fm.bm)}
\CommentTok{\# mult \textless{}{-} qt(0.975, df = rdf)}
\CommentTok{\# }
\CommentTok{\# cis \textless{}{-} se*mult}
\CommentTok{\# fm.bm2 \textless{}{-}lmer(bodymass \textasciitilde{} period + non.breeding.strategy*sex + (1|birdid.fk), data = bodymass1)}
\CommentTok{\# summary(fm.bm2)}
\end{Highlighting}
\end{Shaded}

We found no differences in body mass at the onset of the breeding season
for migratory strategy (coefficient +/- SE: 0.0250851 \(\pm\)
0.0601827), sex (coefficient +/- SE: -0.0534162 \(\pm\) 0.0782538), or
the interaction between the two (coefficient +/- SE: -0.1148465 \(\pm\)
0.0928519) (Fig. 5).

\begin{Shaded}
\begin{Highlighting}[]
\FunctionTok{ggplot}\NormalTok{(bodymass1) }\SpecialCharTok{+}
  \FunctionTok{geom\_boxplot}\NormalTok{(}\FunctionTok{aes}\NormalTok{(}\AttributeTok{x=}\NormalTok{sex, }\AttributeTok{y =}\NormalTok{ bc, }\AttributeTok{fill =}\NormalTok{ non.breeding.strategy)) }\SpecialCharTok{+} 
  \FunctionTok{scale\_fill\_manual}\NormalTok{(}\AttributeTok{values =}\NormalTok{ pal, }\AttributeTok{labels =} \FunctionTok{c}\NormalTok{(}\StringTok{"Residents"}\NormalTok{, }\StringTok{"Migrants"}\NormalTok{),}
                     \AttributeTok{name =} \StringTok{"Migratory Strategy"}\NormalTok{) }\SpecialCharTok{+}
  \FunctionTok{ylab}\NormalTok{(}\StringTok{"Body Condition (Mass/Tarsus Length)"}\NormalTok{) }\SpecialCharTok{+}
  \FunctionTok{xlab}\NormalTok{(}\StringTok{"Sex"}\NormalTok{) }\SpecialCharTok{+}
  \FunctionTok{theme\_minimal}\NormalTok{()}
\end{Highlighting}
\end{Shaded}

\begin{figure}
\centering
\includegraphics{blackbird_POL_files/figure-latex/unnamed-chunk-13-1.pdf}
\caption{Fig. 5. Breeding season body condition (bodymass/tarsus length)
by sex and migratory strategy}
\end{figure}

\textbf{DO WE WANT TO LEAVE THE ANLYSIS HERE OR STILL AIM FOR CONNECTING
TO A DEB?}\\
\emph{Need to consider the ms complexity, timing of the availability of
the DEB extension to \texttt{nichemapr}, target journal, as well as what
exactly we could expect to get from the addition of a DEB etc.}\\
\emph{My \$0.02: I think it would be nice to be able to point
quantitatively to where we expect this to go - for which the DEB might
be nice. That said, I suspect it might be very difficult to sensibly
parameterize the DEB because the costs of migration itself are going to
be very difficult to capture, especially those periods of tissue
hypertrophy and atrophy that pre-/ante-cede migrations themselves and
without those quantitatively account for, I'm not positive the DEB is
going to be any more convincing than reasoning by logic alone\ldots{}
(see the bullet points in the discussion as of now)}

\hypertarget{discussion}{%
\subsection{Discussion}\label{discussion}}

\begin{itemize}
\tightlist
\item
  We observed clear differences in the metabolic dynamics during the
  over wintering period between migrants and year-round residents.

  \begin{itemize}
  \tightlist
  \item
    Differences in the weather context to which individuals were exposed
    during the wintering period resulted in markedly different metabolic
    costs of maintaining body temperature set point, despite
    approximately equivalent observed body temperatures between the two
    strategies.
  \item
    This resulted in substantial differences in the 'metabolic
    remainder''. The metabolic activity of migrants was 2-3X higher than
    that of residents during the wintering period by virtue of milder
    weather conditions.
  \end{itemize}
\item
  Clutch size in Blackbirds is markedly invariant and there is no
  evidence of differential reproductive investment between the two
  strategies \textbf{{[}@Nils please confirm/elaborate - this is based
  on a super quick read of some of the lit{]}}

  \begin{itemize}
  \tightlist
  \item
    Therefore, it is unlikely that this excess metabolic activity is
    being used to increase annual fecundity.
  \item
    Instead, we suggest that this increase metabolic activity is used to
    pay for the costs of migration.
  \item
    While there is debate about the direct costs of migration itself,
    clearly migration requires substantial physiological adaptation
    which are metabolically expensive - this metabolic remainder which
    is enabled by the milder conditions that migration provides, in
    turn, pay for the costs of migrating in the first place.
  \item
    Interestingly, this results in indistinguishable cumulative
    metabolic throughput (though with varying dynamics - ref Nils' in
    prep ms here).
  \end{itemize}
\item
  There are multiple theoretical explanations for the maintenance of
  partial migration as an evolutionary stable strategy within an
  admixing population. (refs to Hannah Kokko work, Taylor and Norris, my
  Ecology paper, etc). All essentially imply different pathways to
  equivalent individual performance (though many invoke density
  dependence to get there).

  \begin{itemize}
  \tightlist
  \item
    Because we assume annual reproductive investment to be equivalent,
    life history theory would then predict roughly equivalent survival
    risks across strategies.
  \item
    Intepretting top line metabolic rate as a rough proxy for pace of
    life, total metabolic rate between residents and migrants in this
    study were indistinguishable, as predicted by theory.
  \item
    However, the metabolic dynamics, and specifically the relative
    allocation to metabolic objectives varied between these strategies
    such that the beneficial conditions offered by migration balance out
    the energetic costs of the behavior to generate equivalent
    individual performance.

    \begin{itemize}
    \tightlist
    \item
      This finding uncovers a potential physiological mechanism by which
      individual heterogeneity in behavior type (migration) can emerge
      even when life histroy strategy does not.
    \item
      Furthermore this suggests a novel explanation for the emergence
      and maintenance of partial migration - namely that the net
      costs-benefits of the two strategies are equivalent while the
      ration between costs/benefits is not. \textbf{{[}make sure this
      really is novel\ldots{]}}
    \end{itemize}
  \end{itemize}
\item
  Something about how this can also help us understand the location of
  seasonal ranges:

  \begin{itemize}
  \tightlist
  \item
    Blackbirds are wintering in a location that exactly allows them to
    balance the cost-benefit equation of migration. If they were able to
    make that net value exceed that of residency then we would expect
    (all else being equal) migration to take over as the dominant or
    exclusive phenotype.
  \item
    Clearly there are other factors influencing this equation,
    especially in other systems where extrinsic mortality is of
    particularly high risk, or migratory routes involving truly risky
    flights (e.g., oceanic crossings).
  \item
    However, that all being said, this is more in keeping with a Winger
    Red Queen type explanation for the emergence and maintenance of
    migration that it is an ``optimal annual routine'' explanation.
  \end{itemize}
\end{itemize}

\hypertarget{acknowledgements}{%
\subsection{Acknowledgements}\label{acknowledgements}}

Thanks for all the fish.

\hypertarget{literature-cited}{%
\subsection{Literature Cited}\label{literature-cited}}

\hypertarget{appendices}{%
\subsection{Appendices}\label{appendices}}

\end{document}
